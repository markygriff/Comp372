% Mark Griffith
3.1-1
The functions f(n) and g(n) are asymptotically non negative, there exists
n0 suchthatf(n)≥0andg(n)≥0foralln≥n0. Thus,wehave that for all n ≥ n0,
f(n)+g(n) ≥ f(n) ≥ 0 and f(n)+g(n) ≥ g(n) ≥ 0. Adding both inequalities
(since the functions are nonnegative), we get f(n) + g(n) ≥ max(f(n),
g(n)) for all n ≥ n0. This proves that max(f(n), g(n)) ≤ c(f(n) + g(n))
for all n ≥ n0 with c = 1, in other words, max(f (n), g(n)) = O(f (n) + g(n)).

4.1-2
\begin{verbatim}
BRUTE FORCE: find every subarray for each element in the array and keep
track of the maximum subarray

  MaxSubarray(array):
    total = 0
    max = 0
    for i from 0 to array.length - 1:
      total = 0
      for y from i to array.length - 1:
        total += array[y]
        if total > max:
          max = total
    return max
\end{verbatim}

4.2-1
Use Strassen's algorithm to compute the matrix product
(1 3) (6 8)
(7 5) (4 2)
Show your work

\begin{align*}
  S1 = 8 - 2 = 6
  S2 = 1 + 3 = 4
  S3 = 7 + 5 = 12
  S4 = 4 - 6 = -2
  S5 = 1 + 5 = 6
  S6 = 6 + 2 = 8
  S7 = 3 - 5 = -2
  S8 = 4 - 2 = 2
  S9 = 1 - 7 = -6
  S10 = 6 + 8 = 14
\end{align*}

\begin{align*}
  P1 = 1 * 6 = 6
  P2 = 4 * 2 = 8
  P3 = 12 * 6 = 72
  P4 = 5 * -2 = -10
  P5 = 6 * 8 = 48
  P6 = -2 * 2 = -4
  P7 = -6 * 14 = -84
\end{align*}

\begin{align*}
  C11 = P5 + P4 - P2 + P6 = 48 - 10 - 8  - 4 = 26
  C12 = P1 + P2 = 6 + 8 = 14
  C21 = P3 + P4 = 72 - 10 = 62
  C22 = P5 + P1 - P3 - P7 = 48 + 6 - 72 + 84 = 66
\end{align}

